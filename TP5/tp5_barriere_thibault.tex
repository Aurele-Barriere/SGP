\documentclass{article}

\usepackage[utf8]{inputenc}
\usepackage[francais]{babel}

\title{Compte Rendu TP5}
\author{Aurèle Barrière \& Jérémy Thibault}
\date{23 novembre 2016}

\def\file#1{\texttt{#1}}
\def\fun#1{\texttt{#1}}
\def\obj#1{\texttt{#1}}

\begin{document}
\maketitle

\section*{Introduction}

Ce TP consistait en l'écrituer de plusieurs fonctions essentielles au bon fonctionnement du système Nachos: les outils de synchronisation (sémaphores, verrous et variables de condition) et la gestion des threads.

\section{Implémentation des outils de synchronisation}
\subsection{Sémaphores}

  Pour commencer, nous avons dû implémenter les fonctions de création, de destruction, ainsi que \fun{P} et \fun{V} des sémaphores, dans le fichier \file{kernel/synch.h}. Nous avons pris soin de désactiver les interruptions dans les fonctions \fun{P} et \fun{V}, puis de restaurer le statut d'interruption précédent.

  Il faut également mettre à jour les champs \obj{ReadyToRun}, \obj{alive} pour indiquer le statut des \obj{threads}.

\subsection{Verrous}

  Les verrous sont très similaires aux sémaphores. Cette fois-ci, en plus des fonctions de création et de destruction, le verrou possède deux méthodes \fun{Acquire} et \fun{Release}. Un thread peut demander à acquérir un verrou si celui est libre. Dans ce cas, le thread peut continuer. Sinon, le thread est mis en attente et éventuellement prendra le contrôle du verrou, s'il est libéré. %chktex 19

  Le verrou possède une liste des processus qui attendent sa libération: quand la fonction \fun{Release} est appellée, il ne faut déverrouiller le verrou que si aucun thread ne l'attend. Sinon, il faut lui attribuer ce thread et le réveiller.

\subsection{Variables de condition}

  Les variables de condition sont là encore semblables aux cas précédents. Un thread peut demander d'attendre sur une variable de condition avec la fonction \fun{Wait}. Ensuite, il est possible d'envoyer un signal à un thread particlier avec \fun{Signal} ou à tous avec \fun{Broadcast}. Les threads signalés peuvent alors reprendre leur déroulement.

\subsection{Code des appels systèmes}

  Pour pouvoir utiliser ces fonctions depuis un programme utilisateur, il est nécessaire de fournir une interface. Celle-ci est définie sous forme d'appels systèmes dans le fichier \file{userlib/syscall.h}.

  Nous avons dû écrire le code de ces appels systèmes. Lors d'un appel système, les arguments sont stockés dans les registres 4, 5, 6 et 7. La valeur de retour est, elle, stockée dans le registre 2.
  Dans le cas général, nous lisons les arguments envoyés en paramètres, puis effectuons les actions nécessaires, et certains tests (par exemple, vérifier que l'objet demandé est bien du type voulu: un sémaphore n'est pas un verrou!). Enfin, nous stockons dans le registre 2 la valeur de retour, selon qu'il y ait eu une erreur ou non.

\section{Implémentation des threads}

La seconde partie de ce TP était dédiée à l'implémentation des threads. En effet, cela est nécessaire pour executer n'importe quel programme.

Plusieurs fonctions sont nécessaires: une fonction de démarrage d'un thread \fun{Start}, une fonction de fin de thread \fun{Finish} et enfin des fonctions de sauvegarde et de restauration de contexte \fun{SaveProcessorState} et \fun{RestoreProcessorState}

La méthode \fun{Start} ajoute le thread au processus actuel, puis initialise son contexte en allouant de la mémoire sur le stack. Enfin, le thread est ajouté à la liste des threads actifs et prêts.
La méthode \fun{Finish} indique que le thread peut être détruit, le retire de la liste des threads actifs puis appelle la function \fun{Sleep}. C'est ensuite le scheduler qui détruit définitivement le thread.

Les deux méthodes \fun{SaveProcessorState} et \fun{RestoreProcessorState} sauvegardent et restaurent le contenu de chaque registre entier ou flottant ainsi que du registre contenant le \emph{condition code}.

\section{Programmes de test}
\subsection{Rendez-vous entre 3 threads}
\subsection{Test des verrous}
\subsection{Test des variables de conditions}

\section{Résultats}


\end{document}
