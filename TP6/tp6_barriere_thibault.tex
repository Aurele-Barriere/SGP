\documentclass{article}

\usepackage[utf8]{inputenc}
\usepackage[francais]{babel}

\title{Compte Rendu TP6}
\author{Aurèle Barrière \& Jérémy Thibault}
\date{23 novembre 2016}

\def\file#1{\texttt{#1}}
\def\fun#1{\texttt{#1}}
\def\obj#1{\texttt{#1}}

\begin{document}
\maketitle

\section{Implémentation des outils de synchronisation}
\subsection{Sémaphores}

  Pour commencer, nous avons dû implémenter les fonctions de création, de destruction, ainsi que \fun{P} et \fun{V} des sémaphores, dans le fichier \file{kernel/synch.h}. Nous avons pris soin de désactiver les interruptions dans les fonctions \fun{P} et \fun{V}, puis de restaurer le statut d'interruption précédent.

Il faut également mettre à jour les champs \obj{ReadyToRun}, \obj{alive} pour indiquer le statut des \obj{threads}.

\subsection{Verrous}
\subsection{Variables de condition}

\section{Implémentation des threads}

\section{Implémentation des exceptions}

\section{Programmes de test}
\subsection{Rendez-vous entre 3 threads}
\subsection{Test des verrous}
\subsection{Test des variables de conditions}

\section{Résultats}


\end{document}